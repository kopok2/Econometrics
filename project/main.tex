\documentclass{article}
\usepackage{graphicx}
\usepackage[utf8]{inputenc}
\usepackage{polski}
\usepackage[backend=biber]{biblatex}
\addbibresource{bib_cite.bib}

\begin{document}

\title{Regresja zmian cen akcji}
\author{Karol Oleszek}

\maketitle
\newpage
\tableofcontents
\newpage

\section{Wstęp}
Przewidywanie zmian cen akcji oraz innych instrumentów finansowych znajduje się w centrum zainteresowania inwestorów. Zmiany cen są podstawowym zjawiskiem powodującym bogacenie się lub ubożenie inwestora indywidualnego bądź instytucjonalnego, dlatego też próby zrozumienia i opisania reguł rządzących tym zjawiskiem są kluczowe dla podejmowania skutecznych decyzji o alokacji kapitału.

Teoria rynków kapitałowyc proponuje wiele różnych wyjaśnień zmienności cen: hipoteza rynku efektywnego \textcite{Bachelier1900} zakłada, że ceny rynkowe akcji w danej chwili odzwierciedlają wszystkie dostępne informacje o spółce; autorzy i zwolennicy hipotezy krótkoterminową zmienność cen opisują jako losowy ruch wokół efektywnej wartości. 

\section{Cel projektu}
To jest cel projektu.

\section{Opis danych}

\section{Dobór zmiennych do modelu}

\section{Wybór postaci modelu}

\section{Weryfikacja statystyczna modelu}

\section{Prognoza}

\section{Interpretacja}

\section{Podsumowanie}

\section{Spis tabel}

\section{Spis rysunków}

\newpage
\section{Literatura}
\printbibliography

\end{document}
