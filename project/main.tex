\documentclass{article}
\usepackage{graphicx}
\usepackage[utf8]{inputenc}
\usepackage{polski}
\usepackage[backend=biber]{biblatex}
\usepackage{indentfirst}
\usepackage{hyperref}
\usepackage{graphicx}
\usepackage{mathrsfs}
\usepackage{amssymb}
\usepackage{amsmath}

\addbibresource{bib_cite.bib}

\begin{document}

\title{Regresja zmian cen akcji}
\author{Karol Oleszek}

\maketitle
\newpage
\tableofcontents

\newpage
\section{Wstęp}
Przewidywanie zmian cen akcji oraz innych instrumentów finansowych znajduje się w centrum zainteresowania inwestorów. Zmiany cen są podstawowym zjawiskiem powodującym bogacenie się lub ubożenie inwestora indywidualnego bądź instytucjonalnego, dlatego też próby zrozumienia i opisania reguł rządzących tym zjawiskiem są kluczowe dla podejmowania skutecznych decyzji o alokacji kapitału.

Teoria rynków kapitałowych proponuje wiele różnych wyjaśnień zmienności cen: hipoteza rynku efektywnego (\textcite{Bachelier1900}) zakłada, że ceny rynkowe akcji w danej chwili odzwierciedlają wszystkie dostępne informacje o spółce; autorzy i zwolennicy hipotezy krótkoterminową zmienność cen opisują jako losowy ruch wokół efektywnej wartości. Hipoteza rynku efektywnego znalazła wielu zwolenników, którzy poddawali w wątpliwość samą zasadność przewidywania cen (\textcite{Cowles1932}), jak również zainspirowała powstanie indeksowych funduszy inwestycyjnych.

Hipoteza rynku efektywngeo spotkała się z szeroką krytyką ze strony ekonomistów i inwestorów giełdowych, którzy wskazywali na kontrprzykłady obalającę hipotezę. Współcześnie właściwie wszystkie duże organizacje finansowe używają różnego rodzaju systematycznych narzędzi do analizy i prognozy zmian cen na rynkach kapitałowych (\textcite{GCM2013}). Duże oraz wciąż rosnące znaczenie ma też algorytmiczny handel (\textcite{Capgemini2012}).

Do złożonego problemu jakim jest symulacja i prognostyka zachowania rynków kapitałowych stosuje się bardzo szeroki wachlarz metod statystycznych, algorytmicznych i ekonometrycznych. Duże zastosowanie mają metody uczenia maszynowego (\textcite{ShenJiangZhang2012}), w tym głebokie sieci neuronowe o niekonwencjonalnych architekturach. Ponadto do prognostyki coraz częściej używa się analizy języka naturalnego (\textcite{WangHoLin2018}).

Poniższa praca zawiera przekrojową regresję zmian cen akcji na rynku amerykańskim w 2019 z wykorzystaniem standardowych narzędzi ekonometrycznych. Zbiór danych służący do konstrukcji modelu zawiera dane z roku 2018, dotyczące sytuacji finansowych, kapitałowych i operacyjnych spółek, zawarte w formie wskaźników i pozycji ze sprawozdań finansowych.

\newpage
\section{Cel projektu}

\subsection{Model wyboru akcji do celów inwestycyjnych}
Celem projektu jest wyznaczenie bazowego poziomu efektywności wyboru spółek, których akcje w nadchodzącym roku zyskają na wartości. Za wybór odpowiadał będzie model, który powstanie przy użyciu metody najmniejszych kwadratów i który będzie mógł służyć jako punkt odniesienia do badania efektywności innych metod predykcji.

Efektywność prognostyczna modelu zostanie zbadana przy użyciu \textit{średniego błędu prognozy ex post}, danego wzorem:

\[ ME = \frac{1}{s} \sum_{t=1}^{s} ( y_t - y_t^P ) \],

Gdzie:

~$s$ - ilość obserwacji w testowym zbiorze danych

~$y_t$ - prawdziwa wartość zmiennej objaśnianej

~$y_t^P$ - prognozowana wartość zmiennej objaśnianej

\medskip

\subsection{Zbadanie zależności pomiędzy zmianą cen, a informacjami finansowymi}
Ponadto model posłuży do oceny wpływu informacji finansowych zawartych w publicznie dostepnych źródłach na przyszłą wartość spółek giełdowych. Ocena ta może być użyteczna przy podejmowaniu decyzji o tym, jakie dane zbierać na temat spółek w celu skutecznego przewidywania ich przyszłej wyceny.

Miarą tej oceny będzie współczynnik determinacji ~$R^2$, dany wzorem:

\[ R^2 = \frac{\sum_{t=1}^{n}(y_t^P-\overline{y})^2}{\sum_{t=1}^{n}(y_t-\overline{y})^2} \],

Gdzie:

~$n$ - ilość obserwacji w uczącym zbiorze danych

~$y_t$ - prawdziwa wartość zmiennej objaśnianej

~$y_t^P$ - prognozowana wartość zmiennej objaśnianej

~$\overline{y}$ - średnia arytmetyczna zmiennej objaśnianej

\newpage
\section{Opis danych}

\subsection{Zbiór danych}
Zbiór danych użyty w projekcie pochodzi z internetowej platformy \href{https://www.kaggle.com/cnic92/200-financial-indicators-of-us-stocks-20142018}{Kaggle} (\textcite{Carbone2019}). Zawiera on zmienną objaśnianą ~$Y$ - procentową zmianę ceny akcji danej spółki w 2019 roku, oraz zmienne objaśniające ~$X_i,  i=1\dots k$ - k-1 wskaźników finansowych i pozycji z formularza \textit{10-K}\footnote{\textit{Form 10-K} jest to coroczne podsumowanie finansowe składane przez amerykańskie spółki giełdowe do \textit{U.S. Securities and Exchange Commission}, federalnej agencji nadzoru finansowego.}, a także zmiennej kategorycznej oznaczającej sektor gospodarki rozważanej spółki.

\subsection{Usuwanie braków danych}
Dane zostały zebrane przy użyciu interfejsu programistycznego \textit{Financial Modeling Prep API} i zawierały pewne braki wynikające z różnic w dokumentach źródłowych. Dla celów analizy usunięte zostały wszystkie obserwacje, w których brakowało więcej niż 50 wartości oraz wszystkie zmienne, w których co najmniej 10\% obserwacji nie miało przypisanej wartości. Po tej transformacji, w zbiorze danych pozostało 4122 obserwacji oraz 179 zmiennych (4392 x 222, 9,98\% braków przed transformacją). Wciąż brakujące 0,82\% wartości zostało zastąpionych średnimi arytmetycznymi odpowiednich zmiennych.

\subsection{Transformacja zmiennej kategorycznej}
Kategoryczna zmienna objaśniająca \textit{Sector}, która przyjmowałą 11 różnych wartości (Consumer Cyclical, Energy, Technology, Industrials, Financial Services, Basic Materials, Communication Services, Consumer Defensive, Healthcare, Real Estate, Utilities) została przekształcona na 10 zmiennych zero-jedynkowych. Po tej operacji zbiór danych składał się ze 188 zmiennych.

\subsection{Zmienne objaśniające}
Zbiór danych składa się z grupy zmiennych opisujących realne wielkości ze sprawozdań finansowych wyrażone w dolarach amerykańskich np. zysk brutto, wydatki na badania i rozwój. Na drugą grupę zmiennych składają się wskażniki finansowe, będące często przekształconymi zmiennymi z pierwszej grupy, wyrażone jako stosunek różnych wielkości np. zysk na akcję, wzrost zysku w ciągu roku. Trzecia grupa zmiennych to przekształcona zmienna \textit{Sector}. Ze względu na liczbę zmiennych pełna lista zmiennych wraz ze statystykami znajduje się w załączniku do pracy.

\newpage
\subsection{Rozkład zmiennej objaśnianej}
Zmienna objaśniana ~$Y$ to wyrażona w procentach zmiana ceny akcji spółki w roku 2019. Przyjmuje ona wartości -99,86\% do +3756,72\%. Większa część wartości jest większa od zera co wskazuje na pozytywną bazową efektywność tzw. strategii \textit{buy and hold}, która polega na zakupie akcji, a następnie oczekiwaniu na wzrost jej wartości.

\begin{table}[h!]
    \begin{center}
    \begin{tabular}{|c | c|} 
    \hline
    Statystyka & Wartość \\
    \hline\hline
    Średnia arytmetyczna & 21,15 \\ 
    \hline
    Odchylenie standardowe & 84,93 \\
    \hline
    Kwartyl dolny & -9,30 \\
    \hline
    Mediana & 17,83 \\
    \hline
    Kwartyl górny & 40,92 \\
    \hline
    Wartość najmniejsza & -99,86 \\
    \hline
    Wartość największa & 3756,71 \\
    \hline
   \end{tabular}
    \end{center}
   \caption{Statystyki - zmienna objaśniana}
\end{table}

\begin{figure}[h!]
    \includegraphics[width=\linewidth]{source/YHistogram.png}
    \caption{Histogram zmiennej objaśnianej}
\end{figure}

\newpage
\subsection{Korelacja}
Korelacja zmiennej objaśnianej ze zmiennymi objaśnianymi jest bardzo słaba, co wskazuje na potencjalnie silnie nieliniowy charakter zachodzących zależności.
\begin{table}[h!]
    \begin{center}
    \begin{tabular}{|c | c|} 
    \hline
    Statystyka & Wartość \\
    \hline\hline
    Średnia arytmetyczna & 0.0006535276131026212 \\ 
    \hline
    Odchylenie standardowe & 0.01430737685985159 \\
    \hline
    Kwartyl dolny & -0.006560508005214499 \\
    \hline
    Mediana & 0.0009029552423272379 \\
    \hline
    Kwartyl górny & 0.008516164727314403 \\
    \hline
    Wartość najmniejsza & -0.07707504123521973 \\
    \hline
    Wartość największa & 0.04058496958769639 \\
    \hline
    \end{tabular}
    \end{center}
   \caption{Korelacja zmiennej objaśnianej ze zmiennymi objaśnianymi}
\end{table}

\begin{figure}[h!]
    \includegraphics[width=\linewidth]{source/CorrHist.png}
    \caption{Histogram korelacji zmiennej objaśnianej ze zmiennymi objaśnianymi}
\end{figure}

\newpage
Korelacja pomiędzy niektórymi zmiennymi objaśniającymi jest silna, co można wyjaśnić zależnościami pomiędzy rozmiarem spółki, a wielkościami w \textit{10-K Form}. Nie wszystkie bazowe zmienne objaśniające mogą znaleźć się w modelu, ponieważ spowodowałoby to wystąpienie zjawiska współliniowości.
\begin{figure}[h!]
    \includegraphics[width=\linewidth]{source/CorrelationMatrix.png}
    \caption{Macierz korelacji}
\end{figure}

\subsection{Podział zbioru danych}
Zbiór danych podzielony jest na dane treningowe(90\%) użyte do estymacji parametrów modelów oraz dane testowe(10\%) służące do oceny prognozy \textit{ex post}.

\newpage
\section{Wybór postaci modelu oraz dobór zmiennych do modelu}
\subsection{Specyfikacja kryteriów}
Praca odpowiada na dwa pytania:
\begin{enumerate}
    \item Jaka jest bazowa efektywność predykcji?
    \item Jaki jest wpływ informacji zawartych w rozważanych danych na zmiany cen akcji?
\end{enumerate}

Odpowiedzią na pierwsze pytanie jest efektywność modelu o najlepszych właściwościach prognostycznych.

Odpowiedzią na drugie pytanie jest ocena współczynnika determinacji poprawnie zweryfikowanego modelu ekonometrycznego.

\subsection{Rozważana klasa funkcji}
Modele wybrane są z klasy funkcji dopuszczalnych ~$\mathscr{F}: \mathbb{R^K\to R}$,
gdzie K: liczba zmiennych w zbiorze danych.

Funkcje przyjmują analityczną postać:

~$\hat{Y}=a_0+a_1*X_1+...+a_k*X_k$;

parametry szacowane są przy użyciu metody najmniejszych kwadratów.

Ze względu na potencjalnie nielinowy charakter zależności rozważony został zbiór danych rozszerzony o przetransformowane zmienne: ~$e^{X_i}, X_i^2, \frac{1}{1+X_i}$ o łącznym rozmiarze 4392 x 888. Dalsze rozszerzanie zbioru danych osłabiło by stabilność numeryczną modeli i nadmiernie zwiększyło by wymiar Wapnika-Czerwonienisa, co osłabiło by możliwości generalizacji przez modele (\textcite{Kaminski2017}).

\newpage
\subsection{Rozważany podzbiór funkcji}
Ze względu na ilość zmiennych, użycie metody o wykładniczej złożoności obliczeniowej, np. metody Hellwiga, byłoby nieefektywne. W związku z tym rozważany jest K-elementowy zbiór funkcji wyłoniony przy użyciu uproszczonej \textit{metody krokowej wstecz}. Rozważany jest model ze wszystkimi zmiennymi, a następnie z modelu usuwana jest najmniej istotna statystycznie zmienna. Procedura jest powtarzana dopóki w modelu jest więcej niż jedna zmienna. Dla celów stworzenia podzbioru funkcji nie jest sprawdzana normalność rozkładu resztowego.

\subsection{Kryterium wyboru modelu prognostycznego}
Spośród rozważanego podzbioru funkcji wybrana jest funkcja z najmniejszym \textit{absolutnym błędem prognozy ex post} oszacowanym przy użyciu sprawdzianu krzyżowego na danych treningowych.

Sprawdzian krzyżowy polega na podziale zboriu danych na J podzbiorów, a następnie wykonaniu J ocen błędu prognozy(za każdym razem inny podzbiór jest uznawany za testowy, J-1 podzbiorów treningowych) i obliczeniu ich średniej.

\[\hat{ME}= \frac{1}{J} \sum_{t=1}^{J} ( ME_i ) \]

\subsection{Kryterium wyboru modelu analitycznego}
Spośród rozważanego podzbioru funkcji wybrana jest funkcja z najwyższym współczynnikiem determinacji ~$R^2$, która została poprawnie zweryfikowana. W przypadku wystąpienia niepożadanych zjawisk w modelu, rozważany podzbiór funkcji powiększony zostaje o model z zastosowanymi stosownymi korektami.


\newpage
\section{Weryfikacja poprawności modelu}
Poniżej opisane są procedury weryfikacji poprawności modelu oszacowanego metodą najmniejszych kwadratów. Przyjęty poziom istotności ~$\alpha=0,05$.

\subsection{Współliniowość}
Wyznaczamy k modeli MNK, gdzie zmienną objaśnianą jest jedna ze zmiennych objaśniających, a zmiennymi objaśniającymi pozostałe zmienne. Wyznaczamy k współczynników determinacji; jeżeli którykolwiek z nich większy jest niż 0,9 , to w modelu występuje niepożądane zjawisko współiniowości zmiennych.

\subsection{Koincydencja}
Jeżeli dla każdego i=1..k:
\[sgn(r_i) = sgn(a_i)\],

gdzie:

~$r_i$: korelacja pomiędzy zmienną objaśnianą, a i-tą zmienną objaśniającą.

~$a_I$: oszacowany i-ty parametr modelu.

to model jest koincydentny. Koincydencja modelu jest pożądaną cechą.

\subsection{Efekt katalizy}
Niech ~$(X_i, X_j)$ będzie regularną parą korelacyjną. Wówczas jeżeli ~$r_{ij} < 0$ lub ~$r_{ij} > \frac{r_i}{r_j}$, gdzie:

~$r_{ij}$: korelacja między i-tą, a j-tą zmienną objaśniającą.

to zmienna ~$X_i$ jest katalizatorem. Poprawny model nie zawiera katalizatorów.

\newpage
\subsection{Normalność rozkładu reszt}
Normalność rozkładu składnika resztowego modelu jest cechą niezbędną do poprawnej interpretacji m.in. testów istotności parametrów modelu. Normalność rozkładu można zbadać przy użyciu testu Jarque-Bera.

~$H_0$: składnik losowy modelu ma rozkład normalny.

~$H_1$: składnik losowy modelu nie ma rozkładu normalnego.

Statystyka testowa ~$JB$ ma rozkład ~$\chi^2$ o dwóch stopniach swobody.

\[JB=\frac{n-k}{6}(A^2+\frac{1}{4}(K-3)^2)\],

gdzie:

Współczynnik skośności:
\[A = \frac{\hat{\mu}_3}{\hat{\sigma}^3} = 
\frac{\frac{1}{n} \sum_{t=1}^{n} ( x_t - \overline{x} )^3}
{(\frac{1}{n} \sum_{t=1}^{n} ( x_t - \overline{x} )^2)^{\frac{3}{2}}}
\]

Kurtoza:
\[K = \frac{\hat{\mu}_4}{\hat{\sigma}^4} = 
\frac{\frac{1}{n} \sum_{t=1}^{n} ( x_t - \overline{x} )^4}
{(\frac{1}{n} \sum_{t=1}^{n} ( x_t - \overline{x} )^2)^{2}}
\]

\subsection{Istotność zmiennych objaśniających}
Zmienne w poprawnym modelu są statystycznie istotne. Do badania istotności zmiennych służy test t-Studenta.

~$H_0$: ~$a_i = 0$.

~$H_1$: ~$a_i \neq 0$.

Statystyka testowa ~$t_{a_i}$ ma rozkład t-Studenta o n-(k+1) stopniach swobody.

\[t_{a_i} = \frac{a_i}{D(a_i)}\]

Dyspersja estymatora i-tego parametru modelu:

\[D(a_i)=\sqrt{d_{ii}}\]

Macierz kowariancji estymatora a:

\[\hat{D}^2(a)=S^2(X^TX)^{-1}\]

Estymator wariancji ~$s^2$ składnika losowego:

\[S^2 = \frac{e^Te}{n - (k + 1)}\]

\newpage
\subsection{Istotność współczynnika determinacji}
Istotność współczynnika determinacji(inaczej istotność wszystkich zmiennych naraz) jest pożądaną cechą modelu i może być zbadana za pomocą testu ~$F$.

~$H_0$: ~$a_1 = a_2 = ... = a_k = 0$.

~$H_1$: ~$a_1 \neq 0 \vee a_2 \neq 0 \vee ... \vee a_k \neq 0$.

Statystyka testowa ~$F$ ma rozkład F-Snedecora-Fishera z ~$r_1 = k$ i ~$r_2 = n - (k + 1)$ stopniami swobody.

\[F = \frac{R^2}{(1-R^2)}\frac{n - (k + 1)}{k}\]

\subsection{Liniowość postaci modelu}
Do badania liniowości modelu ekonometrycznego służy test serii.

~$H_0$: model hipotetyczny jest liniowy.

~$H_1$: model nie jest liniowy.

Statystyka testowa ~$Z$ ma asymptotyczny standardowy rozkład normalny.

~$r$ - liczba serii w wektorze resz modelu(uporządkowanych wg. wartości zmiennej objaśnianej).

~$N_1$ , ~$N_2$ - liczba dodatnich i ujemnych reszt modelu.

\[Z = \frac{r-(\frac{2N_1N_2}{n}+1)}{\sqrt{\frac{2N_1N_2(2N_1N_2-n)}{(n-1)n^2}}}\]

\subsection{Homoskedastyczność}
Homoskedastyczność jest pożądaną cechą modelu. Ze względu na wysoką proporcję ilości zmiennych objaśniających do ilości obserwacji homoskedastyczność najlepiej sprawdzić jest przy użyciu testu Goldfielda-Quandta.

~$H_0$: ~$s_1^2 = s_2^2$. Homoskedastyczność.

~$H_1$: ~$s_2^2 \neq s_2^2$. Heteroskedastyczność.

Statystyka testowa ~$F$ ma rozkład F-Snedecora-Fishera z ~$r_1 = n_1 - (k+1)$ i ~$r_2 = n_2 - (k + 1)$ stopniami swobody.

\[F = \frac{\hat{s}_1^2}{\hat{s}_2^2}\]

Próba podzielona jest na dwa zbiory i badana jest równość wariancji w podpróbach.

\[\hat{s}_1^2 = \frac{{e_1}^T{e_1}}{n_1-(k+1)}\]

\[\hat{s}_2^2 = \frac{{e_2}^T{e_2}}{n_2-(k+1)}\]

\newpage
\subsection{Stabilność parametrów modelu}
Stabilność parametrów modelu, która jest pożądana cechą, może zostać zweryfikowana przy pomocy testu Chowa.

~$H_0$: ~$\alpha = \beta = \gamma$. Paremetry modelu są stabilne.

~$H_1$: Parametry modelu nie są stabilne.

Statystyka testowa ~$F$ ma rozkład F-Snedecora-Fishera z ~$r_1 = k+1$ i ~$r_2 = n - 2(k + 1)$ stopniami swobody.

\[F = \frac{RSK - (RSK_1 + RSK_2)}{RSK_1 + RSK_2}\frac{n - 2(k+1)}{k + 1}\]

~$RSK$ - resztowa suma kwadratów oszacowanego modelu:

\[y_t = \alpha_0 + \alpha_1x_{1t}+...+\alpha_kx_{kt}+\epsilon_t,t=1,2,...,n\]

~$RSK_1$ - resztowa suma kwadratów oszacowanego modelu:

\[y_t = \beta_0 + \beta_1x_{1t}+...+\beta_kx_{kt}+\epsilon_{1t},t=1,2,...,n_1\]

~$RSK_2$ - resztowa suma kwadratów oszacowanego modelu:

\[y_t = \gamma_0 + \gamma_1x_{1t}+...+\gamma_kx_{kt}+\epsilon_{2t},t=n_1+1,n_1+2,...,n\]

\subsection{Autokorelacja składnika losowego}
Autokorelacja składnika losowego jest niepożądaną cechą modelu ekonometrycznego. Występowanie zjawiska autokorelacji pierwszego rzędu można badać przy założeniu, że:

\[e_t = \rho e_{t-1}+\eta_t\]

Służy do tego test mnożnika Lagrange'a autokorelacji składnika losowego.

~$H_0$: ~$\rho=0$. Zjawisko autokorelacji I rzędu nie występuje.

~$H_1$: ~$\rho\neq0$. Zjawisko autokorelacji I rzędu występuje.

Szacowany jest model pomocniczy i obliczany jest jego współczynnik determinacji.

\[e_t=\beta_0+\beta_1x_{1t}+\beta_2x_{2t}+...+\beta_kx_{kt}+\beta_{k+1}e_{t-1}+\mu_t\]

Dla dużej próby (~$n>30$) statystyka testowa ~$LM$ ma rozkład ~$\chi^2$ z jednym stopniem swobody.

\[LM = (n-1)R_{e_t}^2\]

\newpage
\section{Korekty}

\subsection{Korekty stabilności}
W związku z rozważaniem szerokiej klasy funkcji opartych o rozszerzony zbiór danych, a także dużą liczbę zmiennych objaśniających, korekty stabilności postaci modelu i stabilności parametrów nie są używane w projekcie.

\subsection{Korekta heteroskedastyczności i autokorelacji}
Do usunięcia heteroskedastyczności i autokorelacji z modelu ekonometrycznego można użyć uogólnionej metody najmniejszych kwadratów.

Estymator parametrów modelu:

\[a = (X^T\Omega^{-1}X)^{-1}X^T\Omega^{-1}y\]

Estymator wariancji składnika losowego:

\[S^2=\frac{e^T\Omega^{-1}e}{n - (k+1)}\]

Estymator macierzy kowariancji estymatorów:

\[\hat{D}^2(a)=S^2(X^T\Omega^{-1}X)^{-1}\]

Macierz ~$\Omega$ nie jest znana, dlatego do korekt używa się estymatorów:

Korekta heteroskedastyczności dla ~$\sigma^2 = 1$ oraz estymatora ~$\sigma_t^2=e_t^2$:

\begin{equation*}
    \Omega_{h}^{-1}=
    \begin{bmatrix}
        \frac{1}{\sigma_1^2} & 0 & ... & 0 \\
        0 & \frac{1}{\sigma_2^2} & ... & 0 \\
        ... & ... & ... & ... \\
        0 & 0 & ... & \frac{1}{\sigma_n^2}
    \end{bmatrix}
\end{equation*}

Korekta autokorelacji I rzędu dla znanego(wyestymowanego) ~$\rho$:

\begin{equation*}
    \Omega_{a}^{-1}=\frac{1}{1-\rho^2}
    \begin{bmatrix}
        1 & -\rho & ... & 0 \\
        -\rho & 1 + \rho^2 & ... & 0 \\
        ... & ... & ... & ... \\
        0 & 0 & ... & 1 \\
    \end{bmatrix}
\end{equation*}

Korektę łączną zapisać można jako(~$\rho$ wyestymowane po korekcie heteroskedastyczności):

\[\Omega_{ha}^{-1}=\frac{1}{1-\rho^2}\Omega_h^{-1}\Omega_a^{-1}\]

\newpage
\section{Wybrane modele}
\subsection{Wybór modelu}
Dla celów wyboru modelu klasa funkcji została rozszerzona o modele z korektą heteroskedastyczności, korektą autokorelacji oraz korektą łączną, co zwiększyło ilość rozważanych modeli do ~$4K=4*888=3552$. Poprawność każdego rozważanego modelu została zweryfikowana. Pełne sprawozdanie z estymacji i weryfikacji wszystkich modeli znajduje się w załączniku do pracy.

\subsection{Model prognostyczny}
\subsection{Postać modelu}
\[ \hat{Y} = \alpha_0 + \alpha_1X_1 + \alpha_2X_2 + \alpha_3X_3\]\subsection{Wyestymowane parametry modelu}
\[\alpha_0 = -0.1497101818473765\]
\[\alpha_1 = 0.9388879776939478\]
\[\alpha_2 = 0.08354509891505879\]
\[\alpha_3 = 0.058522677703370385\]
\subsection{Wskaźniki jakości modelu}
Współczynnik determinacji ~$R^2 = 0.9961196927344912$

Średni absolutny błąd prognozy \textit{ex post} ~$MAE = 0.15393297715304613$
\subsection{Weryfikacja poprawności modelu}
Weryfikacja poprawności modelu przy poziomie istotności ~$\alpha = 0.05$
\subsubsection{Koincydencja}
\[sgn(\alpha_1) = 1\]
\[sgn(r_1) = 1\]
Koincydencja.
\[sgn(\alpha_2) = 1\]
\[sgn(r_2) = 1\]
Koincydencja.
\[sgn(\alpha_3) = 1\]
\[sgn(r_3) = -1\]
Brak koincydencji.
\subsubsection{Katalizatory}
Zmienna ~$X_3$ jest katalizatorem w parze (~$X_3$, ~$X_1$)
\subsubsection{Współliniowość zmiennych}
Zmienna ~$X_1$ w zależności od reszty zmiennych - ~$R^2 = 0.05194183211424597$.
Nie występuje współliniowość.

Zmienna ~$X_2$ w zależności od reszty zmiennych - ~$R^2 = 0.23089677920262042$.
Nie występuje współliniowość.

Zmienna ~$X_3$ w zależności od reszty zmiennych - ~$R^2 = 0.23534537608084938$.
Nie występuje współliniowość.

\subsubsection{Normalność rozkładu reszt}
\[JB = 0.14543929209237996\]
\[\chi^2_{0.05, 2} = 5.991464547107983\]
Reszty mają rozkład normalny.
\subsubsection{Istotność zmiennych objaśniających}
\[t_{\alpha_1} = -2.5679436530465547\]
\[t_{0.05, 6} = 1.9431802803927818\]
Zmienna ~$X_1$ jest statystycznie istotna.
\[t_{\alpha_2} = 1541.529777440913\]
\[t_{0.05, 6} = 1.9431802803927818\]
Zmienna ~$X_2$ jest statystycznie istotna.
\[t_{\alpha_3} = 54.916346321746126\]
\[t_{0.05, 6} = 1.9431802803927818\]
Zmienna ~$X_3$ jest statystycznie istotna.
\[t_{\alpha_4} = 38.66532085339449\]
\[t_{0.05, 6} = 1.9431802803927818\]
Zmienna ~$X_4$ jest statystycznie istotna.
\subsubsection{Istotność współczynnika determinacji}
\[F = 513.4231000667284\]
\[F_{0.05, 3, 6} = 4.757062663089414\]
Współczynnik determinacji ~$R^2$ jest statystycznie istotny.
\subsubsection{Liniowość postaci modelu}
\[Z = -0.5619514869490162\]
\[k_{0.05, 0, 1} = -0.5619514869490162\]
Postać modelu jest liniowa.
\subsubsection{Stabilność parametrów modelu}
\[F = 1.898250974063121\]
\[F_{0.05, 4, 2} = 19.246794344808947\]
Parametry modelu są stabilne.
\subsubsection{Homoskedastyczność}
\[F = 0.19692985596934573\]
\[F_{0.05, 1, 1} = 161.44763879758827\]
Model jest homoskedastyczny.
\subsubsection{Autokorelacja czynnika losowego I rzędu}
\[LM = 0.34237529810662015\]
\[\chi^2_{0.05, 1} = 3.8414588206941285\]
W modelu nie występuje autokorelacja czynnika losowego I rzędu.
\subsubsection{Wynik weryfikacji poprawności modelu}
8/10 testów poprawności modelu dało wynik pozytywny. Model nie jest poprawny.


\subsection{Model analityczny}


\section{Prognoza}

\section{Interpretacja}

\section{Podsumowanie}

\newpage
\section{Spis tabel}
\renewcommand\listtablename{}
\listoftables


\newpage
\section{Spis rysunków}
\renewcommand\listfigurename{}
\listoffigures


\newpage
\section{Literatura}
\printbibliography[heading=none]

\end{document}
