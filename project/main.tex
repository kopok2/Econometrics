\documentclass{article}
\usepackage{graphicx}
\usepackage[utf8]{inputenc}
\usepackage{polski}
\usepackage[backend=biber]{biblatex}
\addbibresource{bib_cite.bib}

\begin{document}

\title{Regresja zmian cen akcji}
\author{Karol Oleszek}

\maketitle
\newpage
\tableofcontents
\newpage

\section{Wstęp}
Przewidywanie zmian cen akcji oraz innych instrumentów finansowych znajduje się w centrum zainteresowania inwestorów. Zmiany cen są podstawowym zjawiskiem powodującym bogacenie się lub ubożenie inwestora indywidualnego bądź instytucjonalnego, dlatego też próby zrozumienia i opisania reguł rządzących tym zjawiskiem są kluczowe dla podejmowania skutecznych decyzji o alokacji kapitału.

Teoria rynków kapitałowych proponuje wiele różnych wyjaśnień zmienności cen: hipoteza rynku efektywnego (\textcite{Bachelier1900}) zakłada, że ceny rynkowe akcji w danej chwili odzwierciedlają wszystkie dostępne informacje o spółce; autorzy i zwolennicy hipotezy krótkoterminową zmienność cen opisują jako losowy ruch wokół efektywnej wartości. Hipoteza rynku efektywnego znalazła wielu zwolenników, którzy poddawali w wątpliwość samą zasadność przewidywania cen (\textcite{Cowles1932}), jak również zainspirowała powstanie indeksowych funduszy inwestycyjnych.

Hipoteza rynku efektywngeo spotkała się z szeroką krytyką ze strony ekonomistów i inwestorów giełdowych, którzy wskazywali na kontrprzykłady obalającę hipotezę. Współcześnie właściwie wszystkie duże organizacje finansowe używają różnego rodzaju systematycznych narzędzi do analizy i prognozy zmian cen na rynkach kapitałowych (\textcite{GCM2013}). Duże oraz wciąż rosnące znaczenie ma też algorytmiczny handel (\textcite{Capgemini2012}).

Do złożonego problemu jakim jest symulacja i prognostyka zachowania rynków kapitałowych stosuje się bardzo szeroki wachlarz metod statystycznych, algorytmicznych i ekonometrycznych. Duże zastosowanie mają metody uczenia maszynowego (\textcite{ShenJiangZhang2012}), w tym głebokie sieci neuronowe o niekonwencjonalnych architekturach. Ponadto do prognostyki coraz częściej używa się analizy języka naturalnego (\textcite{WangHoLin2018}).

Poniższa praca zawiera przekrojową regresję zmian cen akcji na rynku amerykańskim w 2019 z wykorzystaniem standardowych narzędzi ekonometrycznych. Zbiór danych służący do konstrukcji modelu zawiera dane z roku 2018, dotyczące sytuacji finansowych, kapitałowych i operacyjnych spółek, zawarte w formie wskaźników i pozycji ze sprawozdań finansowych.

\section{Cel projektu}
To jest cel projektu.

\section{Opis danych}

\section{Dobór zmiennych do modelu}

\section{Wybór postaci modelu}

\section{Weryfikacja statystyczna modelu}

\section{Prognoza}

\section{Interpretacja}

\section{Podsumowanie}

\section{Spis tabel}

\section{Spis rysunków}

\newpage
\section{Literatura}
\printbibliography

\end{document}
